% !TeX root = Proposal.tex
% !TeX encoding = UTF-8
% !TeX spellcheck = en_US

%=============================================================================
%
% This proposal will be written as an article
%
%=============================================================================
\documentclass[11pt,a4paper,titlepage]{article}

%opening
\title{Robust optimization and parameter estimation of re-entrant lines and
    queueing networks \\ PhD Research Proposal}
\author{Harold Ship \\ University of Haifa}
%=============================================================================
%
% This part of the preamble is the package importing and setup
%
%=============================================================================


% UTF file input
\usepackage[utf8]{inputenc}
\usepackage[T1]{fontenc}
\usepackage[english]{babel}

% math stuff
\usepackage[fleqn]{amsmath}
\usepackage{amsfonts} % for mathbb
\usepackage{mathtools} % loads amsmath plus other tools
\usepackage{fixmath} % fix greek letters
\usepackage{relsize} % for \mathlarger
\usepackage{bm}

% math theorems and theorem environments
\usepackage{amsthm}

% biblatex for references
\usepackage{csquotes} % ensures biblatex quotes things correctly
\usepackage[backend=biber,style=apa,autocite=inline]{biblatex}
\DeclareLanguageMapping{english}{english-apa}
\addbibresource{../personal_bibliography.bib}

% puts hyperlinks in PDF. Must be loaded before cleveref
\usepackage{hyperref}
\hypersetup{hypertexnames=false}

% better cross-refs like Equations, etc
\usepackage{cleveref}

% Glossary, acronym
\usepackage[acronym,toc,nonumberlist]{glossaries}
\loadglsentries{glossary}
\makeglossaries

% drawings
\usepackage{tikz}
\usetikzlibrary{positioning}% To get more advances positioning options
\usetikzlibrary{arrows}% To get more arrow heads
\usetikzlibrary{shapes.geometric}% To get more shapes
\usetikzlibrary{fit}% To fit shapes inside shapes


%=============================================================================
%
% My macros
%
%=============================================================================
\renewcommand*{\vec}[1]{\ensuremath{\bm{#1}}}%
\newcommand*{\mat}[1]{\ensuremath{\mathrm{#1}}}%
\newcommand*{\transpose}[1]{\ensuremath{#1^{\mathrm{T}}}}%
\newcommand*{\dd}{\ensuremath{\mathop{}\!\mathrm{d}}}%
\newcommand*{\dt}[1]{\ensuremath{#1^{\prime}}}%
\newcommand*{\R}{\ensuremath{\mathop{}\!\mathbb{R}}}%
\newcommand*{\RR}[1]{\ensuremath{\mathop{}\!\mathbb{R}^{#1}}}%
\newcommand*{\optimal}[1]{\ensuremath{#1^*}}%
\newcommand*{\uncertainset}[1]{\ensuremath{\mathcal{#1}}}%
\newcommand*{\defnterm}[1]{\textbf{#1}}%
% sclp math model macros
\newcommand*{\qnet}{\ensuremath{\mathcal{N}}}%
\newcommand*{\serv}{\ensuremath{i}}%
\newcommand*{\nserv}{\ensuremath{I}}%
\newcommand*{\class}{\ensuremath{j}}%
\newcommand*{\nclass}{\ensuremath{J}}%
\newcommand*{\queue}{\ensuremath{k}}%
\newcommand*{\nqueue}{\ensuremath{K}}%

% other macros
\newcommand\needc{{\color{purple}\textit{(citation needed)}}}
\newcommand\tbd{{\color{orange}\textit{TBD}}}

% math theorem environments (must be after cleveref)
\theoremstyle{definition}
\newtheorem{assumption}{Assumption}
\newcommand{\assumptionautorefname}{Assumption}


\begin{document}

% these styles are for tikz drawings in this section
\tikzstyle{server}=[%
rectangle,
minimum height=3cm,
%rounded corners=4pt,
text height=0.75cm,
text depth=.5cm,
text width=1cm,
inner xsep=1em,
inner ysep=1em,
text centered,
draw=black!50
]
\tikzstyle{buffer}=[%
rectangle,
minimum height=1cm,
%rounded corners=4pt,
text height=0.75cm,
text depth=.5cm,
text width=1cm,
text centered,
inner sep=0pt,
draw=red!50,
fill=red!20
]
\tikzstyle{input}=[%
rectangle,
minimum height=1cm,
%rounded corners=4pt,
text height=0.75cm,
text depth=.5cm,
text width=1.5cm,
text centered,
inner sep=0pt,
draw=white!0,
fill=white!0
]
\tikzstyle{output}=[%
rectangle,
minimum height=1cm,
%rounded corners=4pt,
text height=0.75cm,
text depth=.5cm,
text width=1cm,
text centered,
inner sep=0pt,
draw=white!0,
fill=white!0
]
\tikzstyle{task}=[%
rectangle,
minimum height=1cm,
%rounded corners=4pt,
text height=0.75cm,
text depth=.5cm,
text width=1cm,
text centered,
inner sep=0pt,
draw=black!50,
fill=orange!20
]
\tikzstyle{taskflow}=[%
semithick,
blue,
below,
->,
>=stealth
]


\pagenumbering{roman}

%=============================================================================
% Make the title page and other boilerplate
%=============================================================================
\maketitle \clearpage
\tableofcontents \clearpage
\iffalse % remove to enable list of figures
\listoffigures \clearpage
\fi
\iffalse % remove to enable list of tables
\listoftables \clearpage
\fi
\printglossaries \clearpage


\pagenumbering{arabic}
\setcounter{page}{1}


%=============================================================================
% Introduction
%=============================================================================
\section{Introduction}
\label{sec:introduction}

In queueing theory,
a re-entrant line,
or \gls{reqn} is a queueing network where tasks can be processed by servers
in a fixed order.
In particular,
the tasks can be processed by the same servers more than once.
Each server has several buffers.
Tasks arrive at the buffers in a predetermined order \needc.
This is similar to \glspl{mcqn},
in which each server has multiple buffers for processing different classes of
tasks.
However, the \gls{reqn} model only contains a single job class \needc.

The scheduling of \gls{reqn} and \gls{mcqn} systems can be modeled and solved
as optimization problems.
We look at finite time horizon scheduling problems.
We can naively model these problems as discrete tasks arriving with an arrival
process A and a service process S.
However,
the resulting optimization problem is too large to solve \needc.
In order to overcome this issue,
we can approximate the system as a fluid flow problem,
where the stochastic processes are replaced with \tbd{}
under certain \tbd{} conditions.
We can then solve the fluid flow problem using \gls{sclp} \needc,
and apply the solution back to the original discrete network.

It is often the case that the data required for optimization is known with a
degree of uncertainty.
In particular,
the exogenous arrival rate and the service rate of tasks may be known only
approximately.
This uncertainty in these rates can arise from several sources \needc:
\begin{itemize}
    \item they may be difficult to measured exactly,
    \item they may change slightly over time,
    \item there may be changes,
    \item it may not be possible to assign decision variables with
        the precision of the solution,
        causing inaccuracies to accumulate.
\end{itemize}

There are different strategies for dealing with uncertainty in
optimization problems.
\Gls{so} is one such strategy.
The main idea of \gls{so} is to treat the data as random variables.
In our case,
this would mean the arrival rates and service rates would be treated as
random variables.
Usually,
assumptions are made about the family or the distribution of these variables,
and part of the optimization algorithm involves obtaining estimates for them.
\Gls{so} tries to find the most likely optimal solution \needc.
\Gls{ro} is another strategy for solving optimization problems with uncertainty
in the data.
In general,
\gls{ro} uses a more conservative decision procedure than \gls{so}.
The solution obtained,
known as the robust solution,
is a worst-case solution.
This means that it is the best solution that is feasible in for all possible
values of the data,
where possible means within the values allowed \needc.
The set of allowed values for the data is called the \textbf{uncertainty set}.

Recently,
there has been work on \acrlong{ro} of \gls{sclp} problems,
in particular for \gls{mcqn}.
The current state of the art can solve \gls{sclp} for constant data.
However,
there is still a lot of outstanding work.

\begin{figure}
    \centering
    \begin{tikzpicture}[node distance=2.5cm]
        \node[buffer] (B1) at (0,0) {$B_1$};
        \node[buffer] (B2) [below of=B1] {$B_2$};
        \node[buffer] (B3) [right of=B1] {$B_3$};
        \node[buffer] (B4) [right of=B2] {$B_4$};
        \node[input] (I1) [left of=B1] {Type 1};
        \node[input] (I2) [left of=B2] {Type 2};
        \node[output] (O1) [right of=B3] {};
        \node[output] (O2) [right of=B4] {};
        \node[server,fit=(B1) (B2)] (S1) {};
        \node[fill=white] at (S1.north) {Server 1};
        \node[server,fit=(B3) (B4)] (S2) {};
        \node[fill=white] at (S2.north) {Server 2};
        \path[taskflow] (B1) edge node {$\mu_1$} (B3);
        \path[taskflow] (B2) edge node {$\mu_2$} (B4);
        \path[taskflow] (I1) edge node {$\lambda$} (B1);
        \path[taskflow] (I2) edge node {$\lambda$} (B2);
        \path[taskflow] (B3) edge node {$\mu_3$} (O1);
        \path[taskflow] (B4) edge node {$\mu_4$} (O2);
    \end{tikzpicture}
    \caption[Example queueing network with two servers.]{
        \label{fig:queue-net}
        Example queueing network with two servers.
    }
\end{figure}

%=============================================================================
% Theory
%=============================================================================
\section{Theory}
\label{sec:theory}


\subsection{\acrshort{mcqn}}
\label{subsec:theory:mcqn}

Let $\qnet$ be a queueing network containing $i = 1, \ldots, I$
servers operating over a finite time interval $[0,T]$.
The servers process fluid which is divided into classes $j = 1, \ldots, J$,
and fluid from class $j$ processed by one server $s(j) = i$.
Fluid waits for processing in buffers $k = 1, \ldots, K$,
and,
once processed,
flows from buffer $f(j) = k$.
An conceptual example of fluid or task flows can be seen in \Cref{fig:task-flows}.
Tasks entering $\qnet$ from the outside are called exogenous,
and tasks of class $j$ arrive at one of the servers with a rate $a_j$.
We use this constant rate to approximate the Poisson arrival process
with parameter $a_j$.

For fluid of class $j$,
we apply a control to server $s(j)$ at time $t$
to process fluid at rate $u_j(t)$ per unit time,
giving us a vector of control variables $u(t) \in \RR{J}$.
Also at time $t$,
the amount of fluid awaiting processing in the $k$ buffers is $x_(t)$.

The specification of the basic \gls{mcqn} problem is:
\begin{align} \label{eq:mcqn}%
    % looks like 3 columns but is 6 as & makes a right-aligned
    % and left-aligned pair so we have 6 columns but only use the
    % left, right, and left from the 3 pairs
    && \max\limits_{u(t), x(t)} &&& \int\limits_0^T(\gamma^\top +
      (T-t)c^\top)u(t)+h^\top x(t)\mathrm{d}t \nonumber \\
    & \mbox{MCQN} & \mbox{s.t} &&& \int\limits_0^T G u(s)\mathrm{d}s +
%      F x(t)   \leq \alpha+at, \\
      x(t)   \leq \alpha+at, \\
    &&&&& H u(t) \leq b, \nonumber \\
    &&&&& x(t), u(t) \geq 0, ~ 0 \leq t \leq T, \nonumber
\end{align}


$\gamma \in \RR{J}$ is processing cost per job class,
$c$ is \tbd.
$h \in \RR{K}$ is the holding cost per buffer,
$\alpha \in \RR{K}$ is the initial amount of fluid in the buffers
$b \in \RR{I}$ is the maximum processing rate of the $J$ servers.
Additionally,
we have the matrices $G \in \RR{J \times K}$ and $H \in \RR{I \times J}$.
$G$ describes the inflows $(1)$ and outflows ($p_{j,k}$).
$H$ describes the service times,
the reciprocals of the maximum service rates $\mu_j$.

\begin{align}
    G = G_{j,k} = &
    \begin{cases}
        1 & f(j) = k \\
        -p_{j,k} & f(j) != k
    \end{cases} \\
    H = H_{i,j} = &
\begin{cases}
    1/\mu_j & s(j) = i \\
    0 & s(j) != i
\end{cases}
\end{align}

\begin{figure}
    \centering
    \begin{tikzpicture}
        \node[task] (T1) at (0,0) {$T_1$};
        \node[task] (T2) [below left=of T1] {$T_2$};
        \node[task] (T3) [below right=of T1] {$T_3$};
        \node[task] (T4) [below=of T3] {$T_4$};
        \node[task] (T5) [below left=of T4] {$T_5$};
        \path[taskflow] (T1) edge node {} (T2);
        \path[taskflow] (T2) edge node {} (T5);
        \path[taskflow] (T1) edge node {} (T3);
        \path[taskflow] (T3) edge node {} (T4.north);
        \path[taskflow] (T4.south) edge node {} (T5.north);
    \end{tikzpicture}
    \caption[Example fluid or task flows.]{
        \label{fig:task-flows}
        Example fluid or task flows.
        A flow is the set of arrows coming out of a task node.
        Task $T_1$ flows to $T_2$ and $T_3$.
        Task $T_2$ flows straight to $T_5$.
        Task $T_3$ flows to $T_4$,
        which then flows to $T_5$ as well.
    }
\end{figure}

\tbd{} Robust MCQN

\tbd{} Re-entrant

\tbd{} Robust re-entrant

%=============================================================================
% Questions
%=============================================================================
\section{Questions}
\label{sec:questions}

We would like to known the bounds of the arrival and service rates.
If there is historical data,
what is the state of the art on these estimates?

\begin{enumerate}
    \item How can we estimate the parameters and the uncertainty set with and
    without historical data?
    How can we compute the robust counterpart of \gls{sclp} from this?
    How can we create a policy for parameter estimation that doesn't lose too
    much in terms of the objective?
    It may be possible to estimate the data by running a system which is far
    from optimal.
    In many cases this could be expensive.
    \item Can we develop a system and method to generate policies from online
    networks that can detect changes in the data and self-adjust?
    \item Can we extend \gls{sclp} to piecewise constant parameters,
    in particular the task arrival rates?
    \item What other ways can we extend the arrival process?
    \item How can we estimate the uncertainty set for the robust problem?
\end{enumerate}

\printbibliography
\label{sec:bibliography}

\end{document}
