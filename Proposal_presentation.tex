% !TeX root = Proposal_presentation.tex
% !TeX encoding = UTF-8
% !TeX spellcheck = en_US

%=============================================================================
%
% This proposal as a presentation
%
%=============================================================================
\documentclass[pdf]{beamer}

%opening
\title{Optimal Control of Dynamic Workload Placement}
\subtitle{PhD Research Proposal \\ under the supervision of \\ Dr Itai Dattner \\ *Dr Evgeny Shindin}
\author{Harold Ship}
\institute{University of Haifa}
%=============================================================================
%
% This part of the preamble is the package importing and setup
%
%=============================================================================


% UTF file input
\usepackage[utf8]{inputenc}
\usepackage[T1]{fontenc}
\usepackage[english]{babel}

% math stuff
\usepackage{amsmath}
\usepackage{amsfonts} % for mathbb
\usepackage{mathtools} % loads amsmath plus other tools
\usepackage{fixmath} % fix greek letters
\usepackage{relsize} % for \mathlarger
\usepackage{bm}

% math theorems and theorem environments
\usepackage{amsthm}

% biblatex for references
\usepackage{csquotes} % ensures biblatex quotes things correctly
\usepackage[backend=biber,style=apa,autocite=inline]{biblatex}
\DeclareLanguageMapping{english}{english-apa}
\addbibresource{../personal_bibliography.bib}

% puts hyperlinks in PDF. Must be loaded before cleveref
\usepackage{hyperref}
\hypersetup{hypertexnames=false}

% better cross-refs like Equations, etc
\usepackage{cleveref}

% Glossary, acronym
\usepackage[acronym,toc,nonumberlist]{glossaries}
\loadglsentries{glossary}
\makeglossaries

% drawings
\usepackage{tikz}
\usetikzlibrary{positioning}% To get more advances positioning options
\usetikzlibrary{arrows}% To get more arrow heads
\usetikzlibrary{shapes.geometric}% To get more shapes
\usetikzlibrary{fit}% To fit shapes inside shapes


%=============================================================================
%
% My macros
%
%=============================================================================
\renewcommand*{\vec}[1]{\ensuremath{\bm{#1}}}%
\newcommand*{\mat}[1]{\ensuremath{\mathrm{#1}}}%
\newcommand*{\transpose}[1]{\ensuremath{#1^{\mathrm{T}}}}%
\newcommand*{\dd}{\ensuremath{\mathop{}\!\mathrm{d}}}%
\newcommand*{\dt}[1]{\ensuremath{#1^{\prime}}}%
\newcommand*{\R}{\ensuremath{\mathop{}\!\mathbb{R}}}%
\newcommand*{\RR}[1]{\ensuremath{\mathop{}\!\mathbb{R}^{#1}}}%
\newcommand*{\optimal}[1]{\ensuremath{#1^*}}%
\newcommand*{\uncertainset}[1]{\ensuremath{\mathcal{#1}}}%
\newcommand*{\defnterm}[1]{\textbf{#1}}%
% sclp math model macros
\newcommand*{\qnet}{\ensuremath{\mathcal{N}}}%
\newcommand*{\serv}{\ensuremath{i}}%
\newcommand*{\nserv}{\ensuremath{I}}%
\newcommand*{\class}{\ensuremath{j}}%
\newcommand*{\nclass}{\ensuremath{J}}%
\newcommand*{\queue}{\ensuremath{k}}%
\newcommand*{\nqueue}{\ensuremath{K}}%

% other macros
\newcommand\needc{{\color{purple}\textit{(citation needed)}}}
\newcommand\tbd{{\color{orange}\textit{TBD}}}

% math theorem environments (must be after cleveref)
\theoremstyle{definition}
\newtheorem{assumption}{Assumption}
\newcommand{\assumptionautorefname}{Assumption}
\newtheorem{question}{Research Question}


\begin{document}

% these styles are for tikz drawings in this section
\tikzstyle{server}=[%
rectangle,
%minimum height=3cm,
%rounded corners=4pt,
text height=0.75cm,
text depth=.5cm,
text width=1cm,
inner xsep=1em,
inner ysep=1em,
text centered,
draw=black!50
]
\tikzstyle{buffer}=[%
rectangle,
minimum height=1cm,
%rounded corners=4pt,
text height=0.75cm,
text depth=.5cm,
text width=1cm,
text centered,
inner sep=0pt,
draw=red!50,
fill=red!20
]
\tikzstyle{input}=[%
rectangle,
minimum height=1cm,
%rounded corners=4pt,
text height=0.75cm,
text depth=.5cm,
text width=1.5cm,
text centered,
inner sep=0pt,
draw=white!0,
fill=white!0
]
\tikzstyle{output}=[%
rectangle,
minimum height=1cm,
%rounded corners=4pt,
text height=0.75cm,
text depth=.5cm,
text width=1cm,
text centered,
inner sep=0pt,
draw=white!0,
fill=white!0
]
\tikzstyle{task}=[%
rectangle,
minimum height=1cm,
%rounded corners=4pt,
text height=0.75cm,
text depth=.5cm,
text width=1cm,
text centered,
inner sep=0pt,
draw=black!50,
fill=orange!20
]
\tikzstyle{taskflow}=[%
semithick,
blue,
below,
->,
>=stealth
]


% \pagenumbering{roman}

%=============================================================================
% Make the title page
%=============================================================================
\begin{frame}
    \titlepage
\end{frame}

%=============================================================================
% Executive summary
%=============================================================================
\begin{frame}{Executive Summary}
    \begin{itemize}
        \item Agenda
        \begin{itemize}
            \item Motivation
            \item Background
            \item Math Model
            \item Statistical Model
            \item Research Goal
            \item Current Work
        \end{itemize}

        \item Key takeaways
        \begin{itemize}
            \item Research extending \acrshort{sclp}
            \item Developing methods, algorithms and implementations
            \item Preparing PhD research proposal
        \end{itemize}

        \item My expecations
        \begin{itemize}
            \item Feedback on research goal and questions
        \end{itemize}
    \end{itemize}
\end{frame}

%=============================================================================
% Motivation
%=============================================================================
\section{Motivation}
\begin{frame}{Motivation}
    \begin{itemize}
        \item Many applications today need to run as a collection of cooperating services
        running on a set of computer servers.
        \item Often,
        these applications have requirements for processing power,
        geographical location,
        and specialized hardware
        such that several computers are needed to run them.
        \begin{itemize}
            \item For example,
            a support system for autonomous vehicles may consist of computing devices
            in the cloud,
            in fixed stations along the roads,
            and within the vehicles themselves.
        \end{itemize}

        \item The software applications consist of several tasks.
        Some of these may need to run on specific machines.
        However,
        many components can be ``placed'' on any of several machines.
        \item In this research we address the optimization of workload placement of distributed computing tasks.
    \end{itemize}
\end{frame}


%=============================================================================
% Background
%=============================================================================
\begin{frame}{Background}
    Example
    \begin{columns}

    \column{0.7\textwidth}
    \begin{itemize}
        \item Suppose we have an application which provides service to two locations,
        North and South.
        Tasks arrive according to an nonhomogeneous Poisson process,
        with paramters $\lambda_N(t)$ and $\lambda_S(t)$.
        \item Suppose also that there are two server machines,
        Server 1 and Server 2,
        each with CPU capacity $R_1,R_2$ tasks per unit time.
        \item Requests flow from outside to a web service,
        then to a database service,
        then complete.
        We wish to run a web service and a database service
        on each server.
        \item We assign a job class $j$ to each service as indicated in the table and $s(j) = i$ is the associated server.
    \end{itemize}

    \column{0.3\textwidth}
    \begin{table}
        \begin{tabular}{ccc}
            Server & Role & $j$ \\
            \hline
             1 & Web & 1 \\
             2 & Web & 2 \\
             1 & DB & 3 \\
             2 & DB & 4
        \end{tabular}
    \end{table}

    \end{columns}

\end{frame}

\begin{frame}{Math Model}
    \begin{columns}

        \column{0.6\textwidth}
        \begin{itemize}
            \item Each service has a buffer,
            indicated by $B_1, \ldots, B_4$.
            \item We write $a_j(t)$ as the exogenous task input rate of class $j$.
            In this example,
            $a_1(t) = \lambda_N(t)$, $a_2(t) = \lambda_S(t)$,
            $a_3(t) = a_4(t) = 0$.
            This represents the requests to the web servers.
            \item Once a web server processes a request from its buffer,
            it sends the request to a database server.
            We label the rate that completed tasks of class
            $j$ flow into buffer $k$ as
            $g_{j,k}$.
        \end{itemize}

        \column{0.4\textwidth}
        \begin{figure}
            \centering
            \resizebox{\textwidth}{!}{
                \begin{tikzpicture}[node distance=2.5cm]
                    \node[buffer] (B1) at (0,0) {$B_1$};
                    \node[buffer] (B2) [below of=B1] {$B_2$};
                    \node[buffer] (B3) [right of=B1] {$B_3$};
                    \node[buffer] (B4) [right of=B2] {$B_4$};
                    \node[input] (I1) [left of=B1] {};
                    \node[input] (I2) [left of=B2] {};
                    \node[output] (O1) [right of=B3] {};
                    \node[output] (O2) [right of=B4] {};
                    \node[server,fit=(B1) (B3)] (S1) {};
                    \node[fill=white] at (S1.north) {Server 1};
                    \node[server,fit=(B2) (B4)] (S2) {};
                    \node[fill=white] at (S2.south) {Server 2};
                    \path[taskflow] (B1) edge node {$g_{1,3}$} (B3);
                    \path[taskflow,above left] (B1) edge node {$g_{1,4}$} (B4);
                    \path[taskflow, below left] (B2) edge node {$g_{2,3}$} (B3);
                    \path[taskflow] (B2) edge node {$g_{2,4}$} (B4);
                    \path[taskflow] (I1) edge node {$a_1$} (B1);
                    \path[taskflow] (I2) edge node {$a_2$} (B2);
                    %\path[taskflow] (B3) edge node {$\mu_3$} (O1);
                    %\path[taskflow] (B4) edge node {$\mu_4$} (O2);
                \end{tikzpicture}
            }% resizebox
            \caption[Example queueing network with two servers.]{
                \label{fig:queue-net}
                Example queueing network with two servers.
            }
        \end{figure}

    \end{columns}

\end{frame}


\begin{frame}{Model - Variables and Cost function}
        \begin{itemize}
            \item Suppose we are interested in minimizing the total waiting time
            of tasks in buffers in a time interval $[0,T]$.
            \item Let $x = x_j(t), j=1,\ldots, 4$ denote the number of tasks at time $t$ in buffers $B_1, \ldots, B_4$.
            \item Each task has a lifetime $\tau_j$ which is an exponential process,
            meaning the tasks ``complete'' according to a Poisson process with parameter $\mu_j$.
            \item It costs $d_j$ per unit time for a task of class $j$ to remain in the buffer.
            \item $u = u_j, i=j,\ldots,4$ is the CPU percent at time $t$ to work on tasks from buffer $B_1, \ldots, B_4$.
            \item It costs $\gamma_j$ per unit time for server $s(j)$ to process a task of class $j$.
        \end{itemize}

        Cost Function:
        $
        \int\limits_0^T \left(\sum\limits_{j=1}^4 \gamma_j u_j(t) + d_j x_j(t) \right) \textrm{d}t
        $

\end{frame}


\begin{frame}{Model - CPU Constraints}
        Remember $R_1,~R_2$ are the CPU capacities of Servers 1,2

        $$
        u_1(t) + u_3(t) \leq R_1 ~ \forall t \in [0,T]
        $$

        $$
        u_2(t) + u_4(t) \leq R_2 ~ \forall t \in [0,T]
        $$



\end{frame}


\begin{frame}{Model - Buffer Constraints}


$$
x_1(t) = x_1(0) + \int_0^t \left( a_1(s) - g_{1,1} u_1(s) \right) \textrm{d}s
$$

$$
x_2(t) = x_2(0) + \int_0^t \left( a_2(s) - g_{2,2} u_2(s) \right) \textrm{d}s
$$

$$
x_3(t) = x_3(0) + \int_0^t \left( a_3(s) - g_{3,3} u_3(s) + g_{1,4} u_4(s) \right) \textrm{d}s
$$

$$
x_3(t) = x_3(0) + \int_0^t \left( a_4(s) - g_{4,4} u_4(s)  + g_{2,3} u_3(s)\right) \textrm{d}s
$$

\end{frame}

\begin{frame}{\acrshort{sclp}}
        \begin{itemize}
            \item This problem is too large to solve using discrete methods.
            \item We approximate the discrete Poisson processes of each task flow with constant values,
            equal to the respective means.
            \item We can transform our model so that \Gls{sclp} to solve it.
            \item One of our research goals is to extend \gls{sclp} to work with piece-wise constant values for the exogenous task arrival rates.
        \end{itemize}
\end{frame}


%=============================================================================
% Statistical model
%=============================================================================
\begin{frame}{Statistical model}

    In a given distributed system we \textbf{observe} values for:

    \begin{itemize}

        \item the arrival rate $a$ of exogenous tasks of each class over fixed time intervals,
        that is as piece-wise constant values over a time period $[0,T]$,
        and

        \item at least one of:

        \begin{itemize}
            \item the number of tasks in the buffers $x_j$ at times
            $t_1, \ldots, t_N \in [0,T]$ for each class $j$
            \item the task outflow rates $g_{j,k}$ when $j \neq k$.
        \end{itemize}

    \end{itemize}
\end{frame}


%=============================================================================
% Research objective
%=============================================================================
\begin{frame}{Research objective}
    \begin{itemize}
        \item The objective of this research is to analyze Workload Placement problems
        containing uncertainty in the task arrival rate,
        the task flow rates,
        and the objective function.
        \item In particular,
        we plan to develop an iterative methodology for simultaneously  optimizing Workload Placement problems while estimating the parameters of these problems.
    \end{itemize}
\end{frame}

%=============================================================================
% Research questions
%=============================================================================
\begin{frame}{Research questions}

    \begin{question}
        Can we extend our \gls{sclp} algorithm to solve problems with piece-wise constant task arrival rates?
    \end{question}

    \begin{question}
        How can we solve workload placement problems using \gls{sclp} while simultaneously estimating the task arrival rate?
    \end{question}

    \begin{question}
        How can we solve workload placement problems using \gls{sclp} while simultaneously estimating the task outflow rates?
    \end{question}

    \begin{question}
        How can we solve workload placement problems using \gls{sclp} while simultaneously estimating a penalty threshold in the cost function?
    \end{question}

\end{frame}


%=============================================================================
% Work in progress
%=============================================================================
\begin{frame}{Current work}

    Preparing PhD Research Proposal

    \begin{enumerate}
        \item Theory

        \begin{itemize}
            \item \Gls{sclp}
            \item \Gls{ro}
            \item \Gls{so}
            \item Parameter estimation for \Glspl{ode}
            \item Stochastic processes
        \end{itemize}

        \item Implementation
        \begin{itemize}
            \item \Gls{sclp} for workload placement
            \item Monte Carlo simulations of Workload Placement problems
            \item Extensions to \gls{sclp} for piece-wise constant arrival rates
            \item Extensions to \gls{sclp} for penalties in objective function
        \end{itemize}

    \end{enumerate}

\end{frame}


\begin{frame}
    \centering \Huge Thank You!
\end{frame}


\end{document}
