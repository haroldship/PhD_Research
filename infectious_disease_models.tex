% !TeX root = Proposal.tex

\tikzset{compartment/.style={%
        rectangle,
        rounded corners=9pt,
        text height=0.75cm,
        text depth=.5cm,
        text width=1cm,
        text centered,
        draw=black!50,
        fill=black!20
    }}
    
% =========================================================================
% Infectious Disease  Models
% =========================================================================
\section{Infectious Disease Models}
\label{sec:infectious-disease-models}

There are many mathematical models used to understand the progression of infectious diseases.
Compartmental models divide the population into distinct groups,
called compartments.
We look at the number of individuals in each compartment.
In these models,
individuals can move from one compartment to another,
and we are interested in the sizes of the compartments and the rates at which they change.
\cite{bertsimas2014robust}

\subsection{SIR Model}
\label{subsec:infectious-disease-models:sir}

The SIR model is a compartmental model with groups:

\begin{tabular}{p{0.05\linewidth}p{0.8\linewidth}}
    $S$ & the \textbf{susceptible} group,%
    who can acquire the disease\\%
    $I$ & the \textbf{infectious} group,%
    who have the disease and can infect others\\%
    $R$ & the \textbf{recovered} group,%
    who are no longer infectious nor susceptible
\end{tabular}

Individuals can go from $S$ to $I$ and from $I$ to $R$,
as indicated in \Cref{fig:compartment-flow-sir}.

\begin{figure}
    \begin{tikzpicture}
        \node [compartment] (S) at (0,0) {$S$};
        \node [compartment] (I) [right=of S]  {$I$};
        \node [compartment] (R) [right=of I]  {$R$};
        \path[semithick,->] (S) edge node {} (I);
        \path[semithick,->] (I) edge node {} (R);
    \end{tikzpicture}
    \caption{
        \label{fig:compartment-flow-sir}
        Susceptible individuals can become infected.
        Infected can recover.
    }
\end{figure}

We have $S + I + R = N$ where $N$ is the population size.
If we assume that $N$ does not change over time,
then without loss of generality we can assume $N=1$ in order to simplify the model.

\begin{align}
    \dt{S} & =  - \beta S I \\%
    \dt{I} & = \beta S I - \gamma I \\%
    \dt{R} & = \gamma I%
\end{align}

The model above requires the following parameters,
or \textit{data}:

\begin{tabular}{cl}
    $\beta$ & product of the contact rates and transmission probability\\%
    $\gamma$ & recovery rate
\end{tabular}

Note that $1/\gamma$ is the \textit{average infectious period}.

\subsection{SEIR Model}
\label{subsec:infectious-disease-models:seir}

The SEIR model builds on the SIR model described in \Cref{subsec:infectious-disease-models:sir} by adding an additional compartment $E$,
which contains members of the population who have been acquired the disease,
but the disease is latent in them;
that is,
it is in some sort of incubation period.
They are not yet infectious.
The progression from compartment to compartment is shown in \Cref{fig:compartment-flow-seir}.

\begin{figure}
    \begin{tikzpicture}
    \node [compartment] (S) at (0,0) {$S$};
    \node [compartment] (E) [right=of S]  {$E$};
    \node [compartment] (I) [right=of E]  {$I$};
    \node [compartment] (R) [right=of I]  {$R$};
    \path[semithick,->] (S) edge node {} (E);
    \path[semithick,->] (E) edge node {} (I);
    \path[semithick,->] (I) edge node {} (R);
    \end{tikzpicture}
    \caption{
        \label{fig:compartment-flow-seir}
        Susceptible individuals can become exposed.
        Exposed then become infected.
        Infected can recover.
    }
\end{figure}

We have $S + E + I + R = N$ where $N$ is the population size.
If we assume that $N$ does not change over time,
then without loss of generality we can assume $N=1$ in order to simplify the model.

\begin{align}
    \dt{S} & = \mu - (\beta I + \mu) S \\%
    \dt{E} & = \beta S I - (\mu + \sigma) E \\%
    \dt{I} & = \sigma E - (\mu + \gamma) I \\%
    \dt{R} & = \gamma I - \mu R %
\end{align}

The parameters or data of the SEIR model is:

The model above requires the following parameters,
or \textit{data}:

\begin{tabular}{cl}
    $\beta$ & product of the contact rates and transmission probability\\%
    $\gamma$ & recovery rate \\%
    $\sigma$ & latent-to-infectious rate
\end{tabular}

\subsection{Age of Infection Model}
\label{subsec:infectious-disease-models:aoi}

Age of Infection is a discrete time compartmental model.

\subsubsection{Math Model}

\begin{align}
    i(t) \sim \operatorname{Poisson} \left( R(t) \frac{S(t-1)}{N} \sum\limits_{\tau=1}^{T_G} {P_G(\tau)i(t-\tau)} \right) & \label{eq:aoi-i-poisson} \\
    S(t) = S(t - 1) - i(t), ~t=1,\ldots,T \label{eq:aoi-s-recurrence} &
\end{align}

\begin{tabular}{p{0.1\linewidth}p{0.75\linewidth}}
    $N$ & population size \\
    $i(t)$ &  number of newly infected on day $t$.
    Once infected, infectious for $d$ days. \\
    $i(t-\tau)$ & the number of infective people with age-of-infection $\tau$ on day $t$ \\
    $P_{\tau}$ & the probability that infective person with age-of-infection $\tau$ infects a susceptible that they meet \\
    $\beta$ & number of contacts per day \\
    $S(t)$ &  is the number of susceptible individuals on day $t$ \\
    $N$ & is the population size \\
    $P_G$ & the generation-time distribution \\
    $R(t)$ &  the reproductive number on day $t$ \\
\end{tabular}

The initial conditions for the model are $i(0)$ (a vector of length $T_G$) and $S(0)$ (a scalar).

\subsubsection{Observation Model}

\begin{equation}
    Y(t) \sim \operatorname { Poisson }\left(p_{Y} \sum_{\tau=1}^{T_{Y}} P_{Y}(\tau) i(t-\tau)\right)
\end{equation}

\begin{tabular}{p{0.1\linewidth}p{0.75\linewidth}}
    $Y(t)$ & the number of newly observed cases on day $t$ (confirmed / hospitalized / dead). \\
    $i(t)$ & the number of newly infected individuals on day $t$ given by the epidemic model. \\
    $p_Y$ & the probability of an infected case to be observed as $Y$ (detection rate / hospitalization rate / death rate). \\
    $P_Y$ & the time distribution from infection to observation of $Y$ (time to detection / time to hospitalization ...) \\
\end{tabular}

\subsubsection{Identifiability of R(t)}

Under what conditions can we estimate $R(t)$?
In other words,
can we provide necessary and sufficient conditions,
under which we can estimate $R(t)$ from samples taken over time?
In particular,
we want to avoid a situation where two functions $R_1(t), R_2(t)$,
with $R_1(t) \neq R_2(t)$,
do not generate samples $Y_1$, $Y_2$ with the same distribution.
If it is the case that these samples have the same distribution,
we cannot identify which of the functions generated the samples
and is therefore the correct one.

\paragraph{The simplest model}

We begin with the simple system:

\begin{align}
    i(t) & \sim  \operatorname{Poisson} \left( R(t) \frac{S(t-1)}{N} \sum\limits_{\tau=1}^{T_G} {P_G(\tau)i(t-\tau)} \right) ~\text{new cases on day} ~ t & \\
    S(t) & = S(t - 1) - i(t), ~t=1,\ldots,T ~  \text{susceptible on day} ~ t & \\
    Y(t) & \sim \operatorname { Poisson }\left(p_{Y} \sum_{\tau=1}^{T_{Y}} P_{Y}(\tau) i(t-\tau)\right) ~ \text{observed cases on day} ~ t &
\end{align}

Under the following assumptions:

\begin{assumption}
    \label{assm:i-normal}
    $i(t)$ above is large enough that it can be approximated by Normal distribution
\end{assumption}
\begin{assumption}
    \label{assm:pg=1}
    $P_G(0) = 1, P_G(t) = 0, t > 0$
\end{assumption}
\begin{assumption}
    \label{assm:py=1}
    $p_Y = 1, P_Y(0) = 1$, $P_Y(t) = 0, t > 0$ so that $Y(t) = i(t)$
\end{assumption}
\begin{assumption}
    \label{assm:s=N}
    $S(t-1) \approx N$ that is $S(t-1)/N \approx 1$
\end{assumption}

We then derive the following basic model:
\begin{align}
    i(t) & = R(t) i(t-1) + \epsilon_t ~ \text{where} ~ \epsilon_t \sim N(0, R(t)i(t-1)) & \nonumber \\
    S(t) & = S(t - 1) - i(t), ~t=1,\ldots,T & \nonumber \\
    Y(t) & = R(t) i(t-1) + \epsilon_t ~ \text{where} ~ \epsilon_t \sim N(0, R(t)i(t-1))  \label{eq:Y-basic} &
\end{align}

The least squares estimate of \Cref{eq:Y-basic} is:
\begin{equation}
    \label{eq:r-hat-basic}
    \hat{R}(t) = \mathop{\mathrm{argmin}}\limits_{R \in \mathbb{R}} \sum\limits_{t=1}^{T} {\left( Y(t) - R(t)i(t-1) \right)^2}
\end{equation}

If $R(t) = \beta$ is constant, then:
$$\hat{\beta} = \left(X^{\top} X\right)^{-1}X^{\top} Y$$
where:

\begin{align*}
    X & = (i(0), \ldots, i(T-1))^{\top} \text{,} \\
    Y & = (Y(1), \ldots, Y(T))^{\top} \text{,} \\
    \operatorname{Var}(\hat{\beta}) & = \frac{\sum{\sigma^2(t)}}{\sum{i(t)^2}}
\end{align*}

So $\beta$ is identifiable if and only if $(X^{\top}X)$ is of full rank.
Next, consider dividing $[0,T]$ into 2 sub-intervals $[0, T^{*}]$ and $[T_{*}, T]$, with

$$
    R(t) =
    \begin{cases}
        \beta_1 & t \in [0,T^{*}] \\
        \beta_2 & t \in [T^{*}, T]
    \end{cases}
$$
where $T^{*}$ is known.
Then $\beta = (\beta_1, \beta_2)$ is identifiable if each of $\beta_1, \beta_2$ are identifiable as above.
More generally, suppose $R(t) = (\beta_1, \ldots, \beta_k), k \ll T$ on intervals $[0, T_1], \ldots, [T_{k-1}, T_k]$. Then $\beta$ is identifiable if each of the $\beta_j$ are identifiable on $[T_{j-1}, T_j]$.

\paragraph{Relaxing \Cref{assm:pg=1}}

Suppose we relax \Cref{assm:pg=1}, meaning that the generation time of the disease is distributed over two days.
Let us write $P_{G} = (P_1, P_2)$ where $P_j$ is the probability that the generation time is $j$ days.
Note that $P_1, P_2 > 0$ and $P_1 + P_2 = 1$.
Then we have the following problem set:

\begin{align}
    i(t) & = R(t) \left(P_{1} i(t-1)+P_{2} i(t-2) \right)+\epsilon_{t} ~ \text{where} ~ \epsilon_t \sim N(0, \sigma_t^2) & \\
    S(t) & = S(t - 1) - i(t), ~t=1,\ldots,T & \nonumber \\
    Y(t) & = R(t) \left(P_{1} i(t-1)+P_{2} i(t-2) \right)+\epsilon_{t} ~ \text{where} ~ \epsilon_t \sim N(0, \sigma_t^2)  \label{eq:Y-relax-assm-1} &
\end{align}

where $\sigma_t^2 = R(t) \left(P_{1} i(t-1)+P_{2} i(t-2) \right)$.

