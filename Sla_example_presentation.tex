% !TeX root = Sla_example_presentation.tex
% !TeX encoding = UTF-8
% !TeX spellcheck = en_US

%=============================================================================
%
% This proposal as a presentation
%
%=============================================================================
\documentclass[pdf]{beamer}

%opening
\title{Optimal Control of Dynamic Workload Placement}
\subtitle{PhD Research Proposal \\ under the supervision of \\ Dr Itai Dattner \\ *Dr Evgeny Shindin}
\author{Harold Ship}
\institute{University of Haifa}
%=============================================================================
%
% This part of the preamble is the package importing and setup
%
%=============================================================================


% UTF file input
\usepackage[utf8]{inputenc}
\usepackage[T1]{fontenc}
\usepackage[english]{babel}

% math stuff
\usepackage{amsmath}
\usepackage{amsfonts} % for mathbb
\usepackage{mathtools} % loads amsmath plus other tools
\usepackage{fixmath} % fix greek letters
\usepackage{relsize} % for \mathlarger
\usepackage{bm}

% math theorems and theorem environments
\usepackage{amsthm}

% biblatex for references
\usepackage{csquotes} % ensures biblatex quotes things correctly
\usepackage[backend=biber,style=apa,autocite=inline]{biblatex}
\DeclareLanguageMapping{english}{english-apa}
\addbibresource{../personal_bibliography.bib}

% puts hyperlinks in PDF. Must be loaded before cleveref
\usepackage{hyperref}
\hypersetup{hypertexnames=false}

% better cross-refs like Equations, etc
\usepackage{cleveref}

% Glossary, acronym
\usepackage[acronym,toc,nonumberlist]{glossaries}
\loadglsentries{glossary}
\makeglossaries

% drawings
\usepackage{tikz}
\usetikzlibrary{positioning}% To get more advances positioning options
\usetikzlibrary{arrows}% To get more arrow heads
\usetikzlibrary{shapes.geometric}% To get more shapes
\usetikzlibrary{fit}% To fit shapes inside shapes


%=============================================================================
%
% My macros
%
%=============================================================================
\renewcommand*{\vec}[1]{\ensuremath{\bm{#1}}}%
\newcommand*{\mat}[1]{\ensuremath{\mathrm{#1}}}%
\newcommand*{\transpose}[1]{\ensuremath{#1^{\mathrm{T}}}}%
\newcommand*{\dd}{\ensuremath{\mathop{}\!\mathrm{d}}}%
\newcommand*{\dt}[1]{\ensuremath{#1^{\prime}}}%
\newcommand*{\R}{\ensuremath{\mathop{}\!\mathbb{R}}}%
\newcommand*{\RR}[1]{\ensuremath{\mathop{}\!\mathbb{R}^{#1}}}%
\newcommand*{\optimal}[1]{\ensuremath{#1^*}}%
\newcommand*{\uncertainset}[1]{\ensuremath{\mathcal{#1}}}%
\newcommand*{\defnterm}[1]{\textbf{#1}}%
% sclp math model macros
\newcommand*{\qnet}{\ensuremath{\mathcal{N}}}%
\newcommand*{\serv}{\ensuremath{i}}%
\newcommand*{\nserv}{\ensuremath{I}}%
\newcommand*{\class}{\ensuremath{j}}%
\newcommand*{\nclass}{\ensuremath{J}}%
\newcommand*{\queue}{\ensuremath{k}}%
\newcommand*{\nqueue}{\ensuremath{K}}%

% other macros
\newcommand\needc{{\color{purple}\textit{(citation needed)}}}
\newcommand\tbd{{\color{orange}\textit{TBD}}}

% math theorem environments (must be after cleveref)
\theoremstyle{definition}
\newtheorem{assumption}{Assumption}
\newcommand{\assumptionautorefname}{Assumption}
\newtheorem{question}{Research Question}


\begin{document}

% these styles are for tikz drawings in this section
\tikzstyle{server}=[%
rectangle,
%minimum height=3cm,
%rounded corners=4pt,
text height=0.75cm,
text depth=.5cm,
text width=1cm,
inner xsep=1em,
inner ysep=1em,
text centered,
draw=black!50
]
\tikzstyle{buffer}=[%
rectangle,
minimum height=1cm,
%rounded corners=4pt,
text height=0.75cm,
text depth=.5cm,
text width=1cm,
text centered,
inner sep=0pt,
draw=red!50,
fill=red!20
]
\tikzstyle{input}=[%
rectangle,
minimum height=1cm,
%rounded corners=4pt,
text height=0.75cm,
text depth=.5cm,
text width=1.5cm,
text centered,
inner sep=0pt,
draw=white!0,
fill=white!0
]
\tikzstyle{output}=[%
rectangle,
minimum height=1cm,
%rounded corners=4pt,
text height=0.75cm,
text depth=.5cm,
text width=1cm,
text centered,
inner sep=0pt,
draw=white!0,
fill=white!0
]
\tikzstyle{task}=[%
rectangle,
minimum height=1cm,
%rounded corners=4pt,
text height=0.75cm,
text depth=.5cm,
text width=1cm,
text centered,
inner sep=0pt,
draw=black!50,
fill=orange!20
]
\tikzstyle{taskflow}=[%
semithick,
blue,
below,
->,
>=stealth
]


% \pagenumbering{roman}

%=============================================================================
% Make the title page
%=============================================================================
%\begin{frame}
%    \titlepage
%\end{frame}


\begin{frame}

    \begin{columns}
        \column{0.6\textwidth}
        \begin{itemize}
            \item Exogenous task arrivals: $a_1=10$
            \item Buffer costs: $d_1 = d_3 = d_4 = 1$
            \item Penalty cost: $\bar{\gamma}_3 = 2$ for $\bar{u}_3$
            \item $g_{1,3} = g_{1,4} = 1$
            \item Initial buffer values: $x_1(0)=5, x_3(0)=5, x_4(0)=0$
            \item $x_1(t) = 5 + 10t - u_1(t)$
            \item $x_3(t) = 5 + u_1(t) - u_3(t) - \bar{u}_3(t)$
            \item $x_4(t) = u_1(t) - u_4(t)$
            \item Same server: $u_1(t) + u_3(t) = 10$
            \item Cost: $\int x_1(t) + x_3(t) + x_4(t) - 2\bar{u}_3(t) \mathrm{d}t$
            \item Equivalent to: $\int 20 + 10t - 2 u_3(t) - u_4(t) - 3 \bar{u}_3(t) \mathrm{d}t$
        \end{itemize}

        \column{0.4\textwidth}
        \begin{figure}
            \centering
            \resizebox{\textwidth}{!}{
                \begin{tikzpicture}[node distance=2.5cm]
                    \node[buffer] (B1) at (0,0) {$B_1$};
                    \node[buffer,draw=white,fill=white] (B2) [below of=B1] {}; %{$B_2$};
                    \node[buffer] (B3) [right of=B1] {$B_3$};
                    \node[buffer] (B4) [right of=B2] {$B_4$};
                    \node[input] (I1) [left of=B1] {};
                    %                \node[input] (I2) [left of=B2] {};
                    \node[output] (O1) [right of=B3] {};
                    \node[output] (O2) [right of=B4] {};
                    \node[server,fit=(B1) (B3)] (S1) {};
                    \node[fill=white] at (S1.north) {Server 1};
                    \node[server,fit=(B2) (B4)] (S2) {};
                    \node[fill=white] at (S2.south) {Server 2};
                    \path[taskflow] (B1) edge node {$g_{1,3}$} (B3);
                    \path[taskflow,above left] (B1) edge node {$g_{1,4}$} (B4);
                    %                \path[taskflow, below left] (B2) edge node {$g_{2,3}$} (B3);
                    %                \path[taskflow] (B2) edge node {$g_{2,4}$} (B4);
                    \path[taskflow] (I1) edge node {$a_1$} (B1);
                    %                \path[taskflow] (I2) edge node {$a_2$} (B2);
                    %\path[taskflow] (B3) edge node {$\mu_3$} (O1);
                    %\path[taskflow] (B4) edge node {$\mu_4$} (O2);
                \end{tikzpicture}
            }% resizebox
%            \caption[Example queueing network with two servers.]{
%                \label{fig:queue-net}
%                Example queueing network with two servers.
%            }
        \end{figure}

    \end{columns}

\end{frame}



%\begin{frame}
%    \centering \Huge Thank You!
%\end{frame}


\end{document}
